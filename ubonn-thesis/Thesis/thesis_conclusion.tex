%==============================================================================
\chapter{Conclusion}
\label{sec:Conclusion}
%==============================================================================

Top quark physics is one of the most active research fields in of all particle physics. With a yukawa coupling close to unity and a resulting large mass, the top quark offers a great potential to observe physics beyond the Standard Model. The decay of four top quarks produced in the same event is a particularly rare process. Therefore, the cross-section of $t\bar{t}t\bar{t}$ very sensitive to additional BSM contributions. Its rarity makes the identification of $t\bar{t}t\bar{t}$ events particularly challenging and makes the need for more powerful and efficient techniques apparent. One promising perspective are Artificial Neural Networks, which outperform other state-of-the-art machine learning techniques in many applications. \\
The analysis presented in this thesis utilizes Feedforward Neural Networks (FNNs) and Recurrent Neural Networks (RNNs) to identify $t\bar{t}t\bar{t}$ events in the same-sign and multilepton channel. The Neural Networks are trained on ATLAS simulated proton-proton collision data. All signal and background datasets are split into a training, a testing, and a validation dataset to ensure that the obtained separation between signal and background events generalize well to unseen data. 18 input features are selected for the Feedforward Neural Networks. The input features that discriminate $t\bar{t}t\bar{t}$ best from the backgrounds are the number of jets and a b-tagging weight. For Recurrent Neural Networks, 9 input features that contain the raw kinematic information for all particles in the event are combined with the three most discriminating features of the FNN study. \\
An important observation is that the renormalization of event weights and the transformation of the input features to normalized distributions is curtail for the performance of the Deep Neural Networks. The chosen number of trainable parameters in a Deep Neural Network is observed to be pivotal to handle overtraining and achieve the best possible performance. The choice of the activation function and weight initializations improves the area under the ROC curve (AUC) by reducing of overtraining in the $t\bar{t}t\bar{t}$ application. The most impactful hyperparameter for both RNNs and FNNs is the learning rate of the optimization algorithm. In contrast to this, the choice of the optimization algorithm itself has no impact on the peak performance for all trained Deep Neural Networks. Small batch size can increase the training speed and can lead to improved performances for short training times. While being a promising prospect for more complicated weight space topologies, polynomial and cyclic learning rate decay do not significantly improve the performance of Deep Neural Networks for the $t\bar{t}t\bar{t}$ application. \\
The highest AUC reached for FNNs is 0.854 on the testing set and $0.852 \pm 0.005$ on the validation set. The best performing RNN has an AUC of 0.842 on the testing set and $0.838 \pm 0.006$ on the validation set. Therefore, it can be stated that in the considered case, constructing features using physical knowledge is more successful than presenting all information to the Neural Network.

\newpage

Using the high score Neural Network region a significant improvement of the signal efficiency from 1.8 to 2.9 for FNNs and to 2.7 for RNNs is obtained. In both cases, the dominant backgrounds $t\bar{t}W$, $t\bar{t}Z$ and $t\bar{t}H$ are reduced by a factor of over 15. In agreement with the official ATLAS analysis $t\bar{t}t$ is the hardest to distinguishing $t\bar{t}t$ background for the Deep Neural Networks.
Truth level studies of $t\bar{t}t$ and $t\bar{t}t\bar{t}$ showed that the reconstruction of two hadronically decaying top quarks is a promising prospect to discriminate $t\bar{t}t\bar{t}$ from $t\bar{t}t$. The performed $\chi^2$ based reconstruction of two hadronically decaying top quarks identified most quarks into the correct category but still needs further improvements in the future to restore the separations observed in the truth level studies. An auspicious possibility to improve the reconstruction is to identify the $W$ bosons produced in association with the $t\bar{t}t\bar{t}$. \\
The training of both FNNs and RNNs on CPUs and GPUs where investigated as a function of the batch size and number of the trainable parameters in a Neural Network. For FNNs and RNNs, GPUs outperformed CPUs and decreased the training time upto a factor of 4. Therefore, GPU training of Neural Networks will become key for particle physics once the Large Hadron Collider is updated to high Luminosities.