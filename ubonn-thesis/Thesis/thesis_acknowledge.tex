%------------------------------------------------------------------------------
\chapter*{Acknowledgements}
\label{sec:ack}
%------------------------------------------------------------------------------

First and foremost, I would like to thank Prof. Dr. Markus Cristianziani for giving me the opportunity to pick a master thesis with a heavy focus not only on physics but also on deep learning. Throughout my time in his working group, I have learned a lot about physics and Neural Networks. The insights gained on how to efficiently present and phrase sentences will probably stick with me for all my life. I am especially thankful for his constructive and fruitful feedback, which was available at all times. I also thank him for presenting me with the opportunity to participate in the summer school for machine learning in high energy physics 2020.
I would like to thank Prof. Dr. Florian Bernlochner, who was so kind to be the second examiner of my thesis and showed great interest in my research topic during my master Colloquium. \\
I am very thankful to have been part of the ATLAS analysis group at the University of Bonn. Their continuous feedback on my working progress was very constructive and helped me to question my assumptions. \\
A special thanks goes to Prof. Dr. Jochen Dingfelder and all the other professors who introduced me to particle physics and helped me develop my interest in the field. \\
People who work in the background are often forgotten. This is why I would like to emphasize my gratitude for the it-support of both at CERN and locally at the University of Bonn. They always provided fast and professional feedback on all technical problems and questions. \\
For providing me feedback on my writing and all the other help in the last years, I would like to thank my cousin Jennifer Eisermann. \\
Last but definitely not least, I would like to thank O\u{g}ul Oncel, who is the Ph.D. student in the working group of Prof. Dr. Markus Cristianziani. His overwhelmingly kind, careful, and helpful character make not only the a exceptional working colleague but also a great friend, which I learned to appreciated during my master thesis. Our discussions were often lengthy, and we rarely had the same opinion about a physics or computing problem. But at the end of the day, different opinions make great and fruitful discussions. After the end of every discussion, he would ask ``are you annoyed at me?'' to give you the conclusive answer O\u{g}ul: Annoyed is the wrong word thankful a much better one!
